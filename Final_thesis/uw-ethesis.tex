\documentclass[letterpaper,12pt,titlepage,oneside,final]{book}

% For PDF, suitable for double-sided printing, change the PrintVersion variable below
% to "true" and use this \documentclass line instead of the one above:
%\documentclass[letterpaper,12pt,titlepage,openright,twoside,final]{book}

% Some LaTeX commands I define for my own nomenclature.
% If you have to, it's better to change nomenclature once here than in a 
% million places throughout your thesis!
\newcommand{\package}[1]{\textbf{#1}} % package names in bold text
\newcommand{\cmmd}[1]{\textbackslash\texttt{#1}} % command name in tt font 
\newcommand{\href}[1]{#1} % does nothing, but defines the command so the
% print-optimized version will ignore \href tags (redefined by hyperref pkg).
%\newcommand{\texorpdfstring}[2]{#1} % does nothing, but defines the command
% Anything defined here may be redefined by packages added below...

% This package allows if-then-else control structures.
\usepackage{ifthen}
\usepackage{listings}
\usepackage{color}

\definecolor{dkgreen}{rgb}{0,0.6,0}
\definecolor{gray}{rgb}{0.5,0.5,0.5}
\definecolor{mauve}{rgb}{0.58,0,0.82}

\lstset{frame=tb,
	language=Java,
	aboveskip=3mm,
	belowskip=3mm,
	showstringspaces=false,
	columns=flexible,
	basicstyle={\small\ttfamily},
	numbers=none,
	numberstyle=\tiny\color{gray},
	keywordstyle=\color{blue},
	commentstyle=\color{dkgreen},
	stringstyle=\color{mauve},
	breaklines=true,
	breakatwhitespace=true,
	tabsize=3
}
\newboolean{PrintVersion}
\setboolean{PrintVersion}{false} 
% CHANGE THIS VALUE TO "true" as necessary, to improve printed results for hard copies
% by overriding some options of the hyperref package below.

%\usepackage{nomencl} % For a nomenclature (optional; available from ctan.org)
\usepackage{amsmath,amssymb,amstext} % Lots of math symbols and environments
\usepackage[pdftex]{graphicx} % For including graphics N.B. pdftex graphics driver 
\usepackage[acronym,automake]{glossaries}
\usepackage{siunitx}
\usepackage{multirow}
\usepackage{setspace}
\usepackage[paper=a4paper,margin=1in]{geometry}
\usepackage{tikz}

%here declear the acronym
\newacronym{who}{WHO}{World Health Organization}
\newacronym{ncoa}{NCOA}{the National Council of Aging}
\newacronym{mems}{MEMS}{Micro-Electro-Mechanical}
\newacronym{imu}{IMU}{Inertia Measurement Unit}
\newacronym{adc}{ADC}{analog-to-digital}
\newacronym{lsb}{LSB}{Least Significant Bit}
\newacronym{ssf}{SSF}{Sensitivity Scale Factor}
\newacronym{mcu}{MCU}{Micro-Controller Unit}
\newacronym{iot}{IoT}{Internet-of-Things}
\newacronym{us}{U.S.}{United State of America}


\makeglossaries
% Hyperlinks make it very easy to navigate an electronic document.
% In addition, this is where you should specify the thesis title
% and author as they appear in the properties of the PDF document.
% Use the "hyperref" package 
% N.B. HYPERREF MUST BE THE LAST PACKAGE LOADED; ADD ADDITIONAL PKGS ABOVE
\usepackage[pdftex,letterpaper=true,pagebackref=false]{hyperref} % with basic options
% N.B. pagebackref=true provides links back from the References to the body text. This can cause trouble for printing.
\hypersetup{
	plainpages=false,       % needed if Roman numbers in frontpages
	pdfpagelabels=true,     % adds page number as label in Acrobat's page count
	bookmarks=true,         % show bookmarks bar?
	unicode=false,          % non-Latin characters in Acrobat’s bookmarks
	pdftoolbar=true,        % show Acrobat’s toolbar?
	pdfmenubar=true,        % show Acrobat’s menu?
	pdffitwindow=false,     % window fit to page when opened
	pdfstartview={FitH},    % fits the width of the page to the window
	pdftitle={uWaterloo\ LaTeX\ Thesis\ Template},    % title: CHANGE THIS TEXT!
	%    pdfauthor={Author},    % author: CHANGE THIS TEXT! and uncomment this line
	%    pdfsubject={Subject},  % subject: CHANGE THIS TEXT! and uncomment this line
	%    pdfkeywords={keyword1} {key2} {key3}, % list of keywords, and uncomment this line if desired
	pdfnewwindow=true,      % links in new window
	colorlinks=true,        % false: boxed links; true: colored links
	linkcolor=black,         % color of internal links
	citecolor=black,        % color of links to bibliography
	filecolor=magenta,      % color of file links
	urlcolor=cyan           % color of external links
}
\ifthenelse{\boolean{PrintVersion}}{   % for improved print quality, change some hyperref options
	\hypersetup{	% override some previously defined hyperref options
		%    colorlinks,%
		citecolor=black,%
		filecolor=black,%
		linkcolor=black,%
		urlcolor=black}
}{} % end of ifthenelse (no else)

% Setting up the page margins...
% uWaterloo thesis requirements specify a minimum of 1 inch (72pt) margin at the
% top, bottom, and outside page edges and a 1.125 in. (81pt) gutter
% margin (on binding side). While this is not an issue for electronic
% viewing, a PDF may be printed, and so we have the same page layout for
% both printed and electronic versions, we leave the gutter margin in.
% Set margins to minimum permitted by uWaterloo thesis regulations:
\setlength{\marginparwidth}{0pt} % width of margin notes
% N.B. If margin notes are used, you must adjust \textwidth, \marginparwidth
% and \marginparsep so that the space left between the margin notes and page
% edge is less than 15 mm (0.6 in.)
\setlength{\marginparsep}{0pt} % width of space between body text and margin notes
\setlength{\evensidemargin}{0.125in} % Adds 1/8 in. to binding side of all 
% even-numbered pages when the "twoside" printing option is selected
\setlength{\oddsidemargin}{0.125in} % Adds 1/8 in. to the left of all pages
% when "oneside" printing is selected, and to the left of all odd-numbered
% pages when "twoside" printing is selected
\setlength{\textwidth}{6.375in} % assuming US letter paper (8.5 in. x 11 in.) and 
% side margins as above
\raggedbottom
\setcounter{tocdepth}{3}
\setcounter{secnumdepth}{3}
% The following statement specifies the amount of space between
% paragraphs. Other reasonable specifications are \bigskipamount and \smallskipamount.
\setlength{\parskip}{\medskipamount}

% The following statement controls the line spacing.  The default
% spacing corresponds to good typographic conventions and only slight
% changes (e.g., perhaps "1.2"), if any, should be made.
\renewcommand{\baselinestretch}{1} % this is the default line space setting

% By default, each chapter will start on a recto (right-hand side)
% page.  We also force each section of the front pages to start on 
% a recto page by inserting \cleardoublepage commands.
% In many cases, this will require that the verso page be
% blank and, while it should be counted, a page number should not be
% printed.  The following statements ensure a page number is not
% printed on an otherwise blank verso page.
\let\origdoublepage\cleardoublepage
\newcommand{\clearemptydoublepage}{%
	\clearpage{\pagestyle{empty}\origdoublepage}}
\let\cleardoublepage\clearemptydoublepage
%======================================================================5
%   L O G I C A L    D O C U M E N T -- the content of your thesis
%======================================================================
\begin{document}
%delete chapter 
\makeatletter
\def\@makechapterhead#1{%
	{\parindent \z@ \raggedright \normalfont
		\ifnum \c@secnumdepth >\m@ne
		\huge\bfseries \thechapter.\ % <-- Chapter # (without "Chapter")
		\fi
		\interlinepenalty\@M
		#1\par\nobreak% <------------------ Chapter title
		\vskip 10\p@% <------------------ Space between chapter title and first paragraph
}}
% For a large document, it is a good idea to divide your thesis
% into several files, each one containing one chapter.
% To illustrate this idea, the "front pages" (i.e., title page,
% declaration, borrowers' page, abstract, acknowledgements,
% dedication,   of contents, list of tables, list of figures,
% nomenclature) are contained within the file "uw-ethesis-frontpgs.tex" which is
% included into the document by the following statement.
%----------------------------------------------------------------------
% FRONT MATERIAL
%----------------------------------------------------------------------
% T I T L E   P A G E
% -------------------
% Last updated May 24, 2011, by Stephen Carr, IST-Client Services
% The title page is counted as page `i' but we need to suppress the
% page number.  We also don't want any headers or footers.
\pagestyle{empty}
\pagenumbering{roman}

% The contents of the title page are specified in the "titlepage"
% environment.	
\begin{titlepage}
		% Side by side figure
		\begin{figure}
			\begin{minipage}[c]{0.4\linewidth}
			\includegraphics[scale=0.3]{VGU}
			\end{minipage}
			\hfil
				\begin{minipage}[c]{0.2\linewidth}
				\includegraphics[scale=0.25]{FRAUAS}
			\end{minipage}	
		\end{figure}
     	%Text front page
     
        \begin{center}
        \normalsize
        	Frankfurt University of Applied Science\\
        	Department 2: Computer Science and Engineering\\
        	\vspace*{0.1cm}
        	Vietnamese - German University\\
        	Department of Electrical Engineering and Infomation Technology
        \vspace*{1cm}	
        \begin{center}
        	Bachelor Thesis
        \end{center}
        \vspace*{1.3cm}
        \Large
        {\bf DESIGN AND IMPLENTATION OF A WIRELESS FALL DETECTION NETWORK PROTOTYPE USING MEMS SENSORS}

        \vspace*{0.5cm}

        \normalsize
        by \\

        \vspace*{0.5cm}

        \Large
        Ho Ngoc Khang Minh \\
      	\normalsize
      	\vspace{0.2cm}
		Matriculation No.: 1148602
		
        \vspace*{1.5cm}
        \large
       	First Supervisor: Prof. Dr.-Ing. Kira Kastell \\
       	Second Supervisor: Dipl. -Ing Mohammad Reza Mansooji \\
       	\vspace{1.7cm}
        \normalsize
        Submitted in partial fulfillment of the requirements\\
        for the degree of Bachelor Engineering in study program\\
        Electrical Engineering \& Information Technology,\\
        Vietnamese - German University

        \vspace*{3.0cm}
		
        Frankfurt am Main, Germany 2018 \\
	
        
        \end{center}
\end{titlepage}

% The rest of the front pages should contain no headers and be numbered using Roman numerals starting with `ii'
\pagestyle{plain}
\setcounter{page}{2}
\cleardoublepage
 % Ends the current page and causes all figures and tables that have so far appeared in the input to be printed.
% In a two-sided printing style, it also makes the next page a right-hand (odd-numbered) page, producing a blank page if necessary.
 


% D E C L A R A T I O N   P A G E
% -------------------------------
  % The following is the sample Delaration Page as provided by the GSO
  % December 13th, 2006.  It is designed for an electronic thesis.
  
  {
  	\begin{center}
  		\Large
  		\bf Disclamer
  	\end{center}
  	{
  		\setstretch{1.5}
  		\par
  		I hereby declare that this thesis is a product of my own work, unless otherwise
  		referenced. I also declare that all opinions, results, conclusions and recommendations are
  		my own and may not represent the policies or opinions of Vietnamese - German University and Frankfurt University of Applied Science. \par
  	}
  	\vspace{3cm}
  	\hspace{11cm}
  	Ho Ngoc Khang Minh
  	\cleardoublepage
  }	

\cleardoublepage

\pagestyle{plain}
\setcounter{page}{3}

\cleardoublepage % Ends the current page and causes all figures and tables that have so far appeared in the input to be printed.
% In a two-sided printing style, it also makes the next page a right-hand (odd-numbered) page, producing a blank page if necessary.

\begin{center}
\vspace*{\fill}
\copyright 2018 by Ngoc Khang Minh, Ho. All right reserved.
\vspace*{\fill}
\end{center}



\cleardoublepage
%\newpage

% A B S T R A C T
% ---------------
\begin{center}\textbf{Abstract}\end{center}
{
	\setstretch{1.5}
	\par
	Falling is becoming a serious problem to elderly people, often causing unpredictable injuries such as hip fractures, head traumas, etc. More seriously, falls may lead to disability or even death on the victim if assistances from caregivers are not received in time. From this situation, there is a need for a wireless communication network which can automatically detect the fall and send an alarm to the caregivers if there is no safe signal from the victim within 10 minutes. There have been several algorithms such as detection of body orientation after a fall, image processing to detect the fall or applying machine learning techniques (Support Vector Machine (SVM), Markov model) in classifying between falling and other activities of daily living (ADL). However, these complex techniques require a huge amount of computations which can result in the system being overloaded or heavily delayed.\par
	Addressing this problem calls for less computation-intensive techniques while retaining the accuracy and robustness of the system. One such approach is the combination of data from both accelerometer and gyroscope. The focus of this thesis is developing a wireless fall detection network which can combine the data from the above sensors to detect falls and distinguish them from fall-like activities. A comparison on the performance of this network with other existing works is also included to evaluate the robustness of the system.\par
	\vspace{1cm}
	\textit{\textbf{Keywords:} elderly people, fall detection, accelerometer, gyroscope,  wireless communication network}
}
\cleardoublepage
%\newpage

% A C K N O W L E D G E M E N T S
% -------------------------------
\begin{center}\textbf{Acknowledgements}\end{center}
{
	\setstretch{1.5}
	\par
	I would like to express my sincere appreciation to Prof. Dr.-Ing. Kira Kastell, Vice President of Frankfurt University of Applied Science (FRA-UAS) for supervising, supporting and encouraging me during the time I conduct my senior project and final thesis. Also, I would like to thank Mr. Dipl.-Ing Mansooji - communication lab engineer at FRA-UAS, Dr. Vo Bich Hien - lecturer and Mr. Duong Huynh Bao - former control lab engineer at Vietnamese - German University for guiding me on technical and programming aspects. I also would like to thank my family for their supports during my bachelor degree time. This project would not be completed without their grateful guidances and supports. Thanks to their encouragements, I can improve myself, not only on the fundamental background in my major, but also necessary skills for my future career. I would like to mention all of my seniors and classmates who have helped me through this thesis.\par
	\vspace{1cm}
}
\cleardoublepage

%\newpage

% D E D I C A T I O N
% -------------------
%\newpage

% T A B L E   O F   C O N T E N T S
% ---------------------------------
\renewcommand\contentsname{Table of Contents}
\tableofcontents
\cleardoublepage
\phantomsection
%\newpage

% L I S T   O F   T A B L E S
% ---------------------------
\addcontentsline{toc}{chapter}{List of Tables}
\listoftables
\cleardoublepage
\phantomsection		% allows hyperref to link to the correct page
%\newpage

% L I S T   O F   F I G U R E S
% -----------------------------
\addcontentsline{toc}{chapter}{List of Figures}
\listoffigures
\cleardoublepage
\phantomsection		% allows hyperref to link to the correct page
%\newpage

%Acronym section%

% L I S T   O F   S Y M B O L S
% -----------------------------
% To include a Nomenclature section
% \addcontentsline{toc}{chapter}{\textbf{Nomenclature}}
% \renewcommand{\nomname}{Nomenclature}
% \printglossary
% \cleardoublepage
% \phantomsection % allows hyperref to link to the correct page
% \newpage

% Change page numbering back to Arabic numerals
\pagenumbering{arabic}

 
\printglossaries
\setstretch{1.5}
%----------------------------------------------------------------------
% MAIN BODY
%----------------------------------------------------------------------
% Because this is a short document, and to reduce the number of files
% needed for this template, the chapters are not separate
% documents as suggested above, but you get the idea. If they were
% separate documents, they would each start with the \chapter command, i.e, 
% do not contain \documentclass or \begin{document} and \end{document} commands.
%======================================================================
\chapter{Introduction}
%======================================================================

%----------------------------------------------------------------------
\section{Introduction to Fall}
%---------------------------------------------------------------------
Nowadays, fall is becoming a dangerous issue which mostly causes the injuries and even leads to disability or fatal death on human, especially the elderly. According to Hwang \textit{et.at.} \cite{static1}, in the United State, one-third to one-half of elderly people over 65 years old fall at least one time each year  and two-third of them will do so again within the next 6 months. Every 11 seconds, there is an elderly person is treated in the hospital's emergency area due to fall-related injuries and every 19 minutes, one faller die \cite{ncoa}. It is also reported that one out every 200 falls results in a hip fracture on people with age among 65 and 69 and increase to one out of 10 for those aged 85 and more \cite{hip_fracture}. The most profound effect of falling is the loss of functioning associated with the dependency of the elderly for the rest of their life. Besides, a great amount time and money have been spent on the medical treatments for the falls. Approximately \$179 million were used as direct medical costs for treating fatal falls and \$19 billion for non-fatal fall injuries within 2000 \cite{cost_fatal}.

\section{Definition of a Fall}
Before starting a research about falls, it is necessary to understand about the meaning of the term \textit{falls}. According to \gls{who}, a fall is defined as an event which results in a person coming to rest inadvertently on the ground or floor or other lower level with or without loss of consciousness or injury \cite{who}. 

\section{Cause of a Fall}
In recent years, a lot of researches have been done by different groups to find out the causes of falls. There have been different ways to classify the causes of a fall such as age and sex, drugs, cognitive functions, postural control, etc. However, because falling is an unintentional action, many causes usually combine to produce a fall. Therefore, these causes can be divided into two main categories, intrinsic and extrinsic, to ease the complication of research activities.
\subsection{Intrinsic risk factors}
Intrinsic factors are the factors which are within the body. It is also the physical aspect of the body that can cause injuries. Intrinsic factors include physical diseases such as cognitive impairment, postural hypotension, cardiovascular problems, etc.\par
\vspace{0.3cm}
\textbf{\textit{Cognitive Impairment}}\par
It is well recognized that cognitive impairment is the common cause of fall on human, especially the elderly. Due to the research of Prudheim \textit{et al.} in 1981, fallers have in general been found to have a higher prevalence of cognitive impairment than non-fallers \cite{cognitive_1}. It is also reported that human with dementia are approximately 3 times more likely to fall than non-demented \cite{cognitive_2}. The reason why patients with cognitive impairment are likely to fall is that they have increased reaction time, increased postural sway and increased leaning balance which result in the decrease on muscle strength, worse balance and poorer mobility. \par
\vspace{0.3cm}
\textbf{\textit{Postural Control}}\par
Postural control is a complex motor skill that requires interaction of multiple body systems which results in the ability to maintain postural orientation or postural stability \cite{postural_1}\cite{postural_2}. The impairment of postural control is indicated as one of significant causes of the fall during any activities at any age. Impaired control of gait and balance are two main aspects of the postural control that have been considered in many studies about falls. Subjectively assessed gait has been reported to be abnormal in many fall activities and other studies using more objective measurements have found some relations between the impaired gait and balance and risk of falling \cite{bibli_book}.

\vspace{0.3cm}
\textbf{\textit{Cardiovascular problem}}\par
Cardiovascular disorders are responsible for approximately 77\% of patients with unexplained or recurrent falls and falls associated with unexplained loss of consciousness \cite{cardio_1}. There is a fact that fallers with cardiovascular disorders have a greater mortality than those with non-cardiovascular or unknown causes \cite{cardio_2}.   
\subsection{Extrinsic risk factors}
Extrinsic factors are those related to the environment such as lighting, walking surface, loose carpets, and high or narrow steps. 

\vspace{0.3cm}
\textbf{\textit{Dim lighting or glare}}\par
Dim lighting is one significant cause of fall on human, especially on the elderly. While walking on the low light condition, the patient's visibility is reduced which prevents them from detecting the obstacles on the walking path and causing stumble and fall. Too bright lighting can also cause fall on human because of creating glares and distorting the way object look. \par
\vspace{0.3cm}
\textbf{\textit{Bad staircase design}}\par

Bad staircase design is also another extrinsic risk that cause the fall on human. Too high or too low rise of staircases have caused a number of falls because people fail to perceive the abnormal elevation change or incur a misstep on descent. Besides, slippery surface of the treads can cause strip on patients while walking up or down the stairs. One more mistake of staircase design that can lead to the fall is lighting on the stair. Poor visibility and inadequate lighting can cause a user to misread the stair edge, resulting in faulty foot placement and falls.
\section{Consequence of Falls}
The consequences of falls are serious to human at any age in any circumstance, but to the elderly, they have significance beyond that in younger people. There are different consequences related to the falls on human including physical effects, psychological effects and other consequences such as dependency, hospital admission or economic consequences. 

\vspace{0.3cm}
\textbf{\textit{Physical consequences}}\par

Physical consequences of falls are always a significant aspect that many scientists have focused on in recent years. According to the \gls{ncoa} of the U.S., falls are the leading cause of fatal injuries and the most common cause of non-fatal traumas \cite{ncoa}. There are several physical injuries such as bruise, fracture or head injury reported to happen with patients who suffer from falls. In such injuries, hip fractures and head traumas are two common problems which usually analyzed by researchers. \par
Hip fractures cause the greatest health problems and greatest number of death. It is reported that a quarter million hip fractures occur each year among people older than 50 years in the U.S. but more common in woman than men and increase in frequency with increasing age \cite{bibli_book} \cite{medicinenet_fracture}. Most patients with the hip fractures after falls are hospitalized, but about half of them cannot return home or live independently after that. In 1986, it costs for more than \$3 billions for direct medical treatment of hip fractures \cite{medicinenet_fracture}.\par 

Head injuries also usually happen to patients after suffering a fall. It is reported that people with the age of over 65 years old make up for 10-15\% of admission to the hospital because of head injuries while three-quarters of them are caused by fall \cite{bibli_book}. In California from 1996 to 1999, 71\% of fall-related head injuries occurred in adults aged 65 and more \cite{head_injury_1}. The head trauma from previous fall can increase the risk of repeated fall and head injuries in the near future.  \par

Finally, the most serious impacts of fall is fatal injuries causing death to patient. According to data from the \gls{who}, the fall morality rate of people with the age of over 65 in the \gls{us} (in 2003) was 36.8 per 100 000 citizens \cite{death_in_USA}. The main reason of these accidental death is that many patients have lie on the ground for several hours before receiving the help from caregivers. \par

\vspace{0.3cm}
\textbf{\textit{Psychological consequences}}\par
Fall can result in long lasting psychological impact, called fear of falling again, more than just short term injuries. For those people who have fallen one time before seem to have a feeling of vulnerability which limits them from their normal activities. They may be worried about falling and hurting themselves again, thus stop going out on their own or reduce their level of physical activities. Unfortunately, in attempting to reduce their risk of falling,  they may increase risk of falling because of the attenuation of physical functions and mobility. 	
\cleardoublepage
\vspace{0.3cm}
\textbf{\textit{Other consequences}}\par
Some research indicated that there are other consequences such as dependency, hospital admission or economic consequence. As already mentioned, the fear of falling again prevent patients from living alone and start depending on the long-term nursing care. This consequence directly leads to increased dependency, which require more time and financial resources from both government and family. \par
Falls are also the main reason for fallers to be admitted to hospital as an emergency case. Due to the information provided by Accident Department at a sample of hospitals in England and Wales, in 1998, approximately 200 000 people over 60 are treated at this hospital per year because of a fall at home \cite{other_consequence_1}.\par
Finally, falls are also a economic burden of every countries around the world, especially for low and lower middle countries. According to the U.S. Center for Disease Control and Prevention, in 2015, there was \$50 billion being used for fall injuries and expected to increase as the population ages and may reach \$67.7 billion by 2020 \cite{ncoa}. These costs are spent for hospitalization, medical, pharmaceutical, nursing home and other costs which related to the medical treatments for fall patients.

\section{Challenges in Detecting Falls}
As mentioned in previous section, most of patient's death is caused by the late assistance from caregivers after fall accidents. Timely detection can minimize the negative impacts of falls on patients. However, depend on age and physical condition, different people have different gaits, different balance ability that result in several types of falls. These falls can happen in any direction at any speed. Therefore, it is still a challenge to design a system which can detect successfully all the falls of patients. \par
Another challenge in detecting the fall is that it is difficult for a system to distinguish a real fall from other fall-likes activities such as sitting, jumping or running, which also result in high acceleration. Therefore, it is required to collect a great amount of data from different subjects; and, then, there should be some analysis to find out the most common pattern of every fall. From that, researchers can design an algorithm to detect falls based on this pattern. However, it is not easy to collect real data for the system in the daily life on real human body, since it is an accidental, dangerous action which can happen randomly at any time. Therefore, it is a overwhelming task for researcher to built a dataset for their system. 

\section{Objectives}
From above challenges, it is necessary to study and design a wireless sensor network which can automatically detect the fall of patient and send alarm to the caregiver instantly to reduce their time of arrival, and hence reduce mortality rate. The sensor node is portable and attached on the chest of user. This sensor node contains one accelerometer and one gyroscope to provide information about the body's acceleration and angular velocity during the fall event. This sensor node can connect to the host computer via Wi-Fi with the help of Wi-Fi module called ESP8266 from ESPRESSIF\textregistered company and send data directly to a monitoring software on the computer. Upon detecting a fall, sensor node will wait for the safe signal from patient within next 10 minutes. If there is no response, an emergency email will be sent immediately to the caregiver. There is also a super bright red led on the sensor node to indicate the fall and help caregiver to easily find out the patient. 
\section{Thesis Organization}
In chapter 1, basic knowledge about falls as well as the causes and consequences of them are introduced. It is also provides the information about current challenges in detecting fall and motivation for this thesis. In chapter 2, a review of existing commercial fall detection devices in the market and researches related to this topic from  other research group is included. Chapter 3 describes the problem of fall detection and method to solve this problem. Also in this chapter is the description of algorithm used in this thesis. Chapter 4 describes the system with both hardware and software aspects. Chapter 5 and 6 describe the data collection and analysis procedure as well as result assessment of the experiments on real human body. Finally, chapter 7 gives the conclusion for this thesis, limitation of current solution and recommendation for future works.
%======================================================================
\chapter{Literature Review}
Nowadays, there are a lot of companies focusing on developing portable fall detection devices to suffice the need of current society where there are more and more elderly people fall every day. These products are used widely by many people around the world and have positive effects on health-care systems of these countries. Beside that, numerous universities and research groups carried out studies on the topic of fall detection from various perspectives and have great contributions to the evolution of health monitoring services.  
\section{Existing Commercial Devices}
One popular fall detection device, which is highly evaluated by customers, is the Medical Fall Alert in the \textit{Home Guardian} option from Medical Guardian\textregistered. It is designed in form of a lightweight wearable neck pendant. This pendant is waterproof and small enough that have no disturbance on user's daily activity. Medical Guadian\textregistered provides a home system including one base unit connecting to the monitoring station through landline and one auto-alert pendant on user body. Whenever detecting a fall, the wearable pendant will send a signal to the base unit and the base station will announce \textit{"Fall Detected, Press or Hold Button to Cancel"}. If the patient actual need help, do not press the button. Then, the system will connect to operators at the monitoring center, they will ask if patient need help or not. If the patient cannot speak because of unconsciousness, the operators will automatically send emergency services to patient's home. The monthly fee for this service is around \$44.95 per month and it is used inside the \gls{us}.

 \begin{figure}[h!]
	\centering
	\includegraphics[scale=0.7]{medical-guardian-fall-alert-system}
	\caption{Medical Guardian Fall Alert System}
\end{figure}
  
Another device on the market is myHalo\texttrademark from MobileHelp\texttrademark. This device has more advantages compared to Medical Guadian one because of full body monitoring functions. It can detect a fall, mornitoring heart rate, skin temperature, sleep/wake pattern, etc. and send alert signal to the authorized contacts. This product supports the medical alert outside the home powered by nationwide cellular network and do not require landline phones. It also provides precise location of patients when they go outside. This enable emergency call center to pinpoint patients' location and direct medical support to them. This service costs from \$41.95 per month and support 24/7. However, because this system offers so many features, it seems to less focus on the fall detection. 

 \begin{figure}[h!]
	\centering
	\includegraphics[scale=0.5]{mobile-help}
	\caption{myHalo Medical Alert System from MobileHelp\texttrademark}
\end{figure}

\section{Existing Products from Other Research Groups}
Beside many commercial fall detection devices being available on the market, there are still many products from different research groups surrounding the fall detection topic. A group of Qiang Li and colleagues have used TEMPO (Technology-Enabled Medical Precision Observation) 3.0 sensor node for their project \cite{li}. This sensor node contains one tri-axial accelerometer and one tri-axial gyroscope and controlled by an TI MSP430F1611 microcontroller. This group proposed an algorithm which combines the data from both accelerometer and gyroscope to reduce both false positive and false negative detection and improving the fall detection accuracy. This system is low computational requirements and real-time response. The authors stated that their method has difficulties in differentiating jumping into bed and falling against wall with a seated posture. 

 \begin{figure}[h!]
	\centering
	\includegraphics{tempo}
	\caption{TEMPO 3.0 sensor node}
\end{figure}

Binh Nguyen and Jonathan Tomkun designed and created a fall detection system for the elderly as a wearable monitoring device which can distinguish between fall and non-fall events  \cite{binh}. This device can link wirelessly with a pre-programmed laptop computer or Bluetooth-compitable mobile phone. Upon detecting a fall, the device communicates wirelessly with the laptop or cellphone to call 911 or issued emergency contacts. This device can also detect abnormal tilt and warns the user to correct their posture to minimize the risk of falling. In addition to visual LED to alert the fall, there are also audio and tacticle alert options for people with hearing or seeing disabilities. Regarding the performance of system, some actions have not been distinguished by one of four proposed algorithm to be falls or non-falls. 

 \begin{figure}[h!]
	\centering
	\includegraphics[scale=0.7]{binhnguyen_device}
	\caption{The final design of device from group research of Binh Nguyen and Jonathan Tomkun}
\end{figure}  

\chapter{Statement of Problem and Methodology}

\section{Problem}
Nowadays, there have been various research groups studying on the academic area about fall detection on human. The majority of them focus on designing a new algorithm to successfully distinguish between fall and fall-like activities. From the first time, most of research groups used the algorithm mainly based on the data from only one accelerometer such as \cite{only_accel_1}\cite{only_accel_2}\cite{only_accel_3}. However, focusing only on the data from the accelerometer can result in many false positives as other activities such as sitting, running and jumping which may also cause large peak accelerations. After that, there are other algorithms rely on the detection of body orientation after the fall such as \cite{body_orientation_1}\cite{body_orientation_2}. The main drawback of these strategies is that the fall can be confused by activities with similar postures such as sleeping, reclining, etc. It is also less effective when a person's fall posture is not horizontal. Then, there are some groups have nominated such complex algorithms that used Support Vector Machine (SVM)\cite{SVM} and Markov model \cite{markov} to detect the fall. These techniques requires huge amount of computational resources which cause the delay on system which effect the robustness and accuracy of the system. 	
\section{Methodology and Algorithm}
From the disadvantages of prior works as mentioned above, this project proposes using an algorithm which combines data from one accelerometer and one gyroscope to detect the fall. The data from accelerometer provides valuable information regarding body inertial change due to impact, while the gyroscope's data provides information about the body's rotational velocity during a fall event. This method helps improving the accuracy of falling detection system without the need of complex computations.  \par
This algorithm is nominated by Quoc T. Huynh \textit{et al.} \cite{main_quoc}, which used data from both accelerometer and gyroscope sensors with predefined critical thresholds, to detect a fall with maximum sensitivity and specificity. Two parameters used to analyze in this algorithm are normalized acceleration and normalized angular velocity. \par

Normalized acceleration is the total sum acceleration vector, named $\boldsymbol{Acc_{norm}}$ containing both static and dynamic acceleration components in different directions. This parameter is calculated with the formula below 

\begin{equation}
\boldsymbol{Acc_{norm}}=\sqrt{(A_x)^2+(A_y)^2+(A_z)^2}
\end{equation}
where $A_x$, $A_y$ and $A_z$ represent the acceleration of three directions $\textit{x, y, z}$. \par 
Normalized angular velocity is the total sum angular velocity vector, named $\boldsymbol{\omega_{norm}}$, containing the rotational velocity components in different direction. This parameter is calculated with the formula below

\begin{equation}
\boldsymbol{\omega_{norm}}=\sqrt{(\omega_{x})^2+(\omega_{y})^2+(\omega_{z})^2}
\end{equation}
where $\omega_x$, $\omega_y$ and $\omega_z$ represent the angular velocity of three directions \textit{x, y, z}.

When stay in stationary condition, the body of user gives the acceleration magnitude of a constant at +1g and angular velocity at $0^{o}/s$. When user falls, there is a rapid change in body's acceleration while angular velocity produces a variety of signals in the fall direction. To detect the fall, the acceleration and angular velocity are compared with critical thresholds. These thresholds are defined as follows by Quoc T. Huynh \textit{et al.} \par 
\vspace{1cm}
{\addtolength{\leftskip}{2cm}
(a) \textit{Lower fall threshold(LFT)}: local minima for the Acc
resultant of each recorded activity are referred to
as the signal lower peak values (LPVs). The $LFT_{acc}$
for the acceleration signals is set at the level of the
smallest magnitude lower fall peak (LFP) recorded.\par
}
{\addtolength{\leftskip}{2cm}
	(b) \textit{Upper Fall Threshold(UFT)}: local maxima for the Acc and $\omega$
	resultant of each recorded activity are referred to as
	the signal upper peak values ($UPV_{s}$). The UFT for
	each of the acceleration ($UFT_{acc}$) and the angular
	velocity ($UFT_{gyro}$) signals are set at the level of the
	lowest upper fall peak (UFP) recorded for the $\boldsymbol{Acc}$
	and $\boldsymbol{\omega}$, respectively. The $UFT_{acc}$ is related to the peak
	impact force experienced by the body segment during
	the impact phase of the fall.\par
}

When a fall happens, the normalized acceleration first falls below the $LFT_{Acc}$ threshold
which indicates the start of a fall event. In the next 0.5 seconds, usually called fall windows,
data from both accelerometer and gyroscope are compared to $UFT_{Acc}$ and $UFT_{Gyro}$. The fall windows was obtained from the literature \cite{window_1}\cite{window_2}. If
the magnitude of both parameters exceed the two upper thresholds, a fall is detected.
However, if only one of these conditions is not satisfied, there is no fall detection. Quoc T.
Huynh et al. \cite{main_quoc} have proven, with practical experiments on real subjects, that the three
conditions above just happen simultaneously only with a fall but not other activities like
standing, running, jumping of sitting up. Figure 3.1 summarizes the main steps of this
algorithm.
\begin{figure*}
	\centering
	\includegraphics[scale=0.8]{algorithm_diagram}
	\caption{Fall detection algorithm}
\end{figure*}
\cleardoublepage

\chapter{System overview}

\section{Hardware Components}

\subsection{Sensor}
Since the algorithm requires both information about acceleration and angular velocity, an \gls{imu} named MPU6050 from Invensense\textregistered, which contains one accelerometer and one gyroscope, has been chosen for its convenience. This sensor can also accepts inputs from an external 3-axis compass via I$^{2}$C sensor bus to provide a complete 9-axis MotionFusion$^{TM}$ output.
 \begin{figure}[h]
	\centering
	\includegraphics[scale=0.25]{mpu6050}
	\caption{MPU-6050 by Invensense \cite{mpu}}
\end{figure}\\
The accelerometer inside MPU6050 board can measure the acceleration in four different programmable full-scale ranges ($\pm2g$, $\pm4g$, $\pm8g$ and $\pm16g$). An integrated 16-bit \gls{adc} supports simultaneous sampling of the accelerometer while requiring no external multiplexer. This accelerometer normally operates at a very small current of \SI{500}{\micro\ampere} with low power consumption. 
According to the data-sheet given by Invensense\textregistered, the raw output of the accelerometer is displayed in form of LSB/g. Divided with an appropriate \gls{ssf}, we can obtain the acceleration in g - standard gravity unit, which is used in this project. Each full scale has its own SFF. Table 2.1 shows the different full-scales of the accelerometer and table 2.2 shows the \gls{ssf} for each scale.
\cleardoublepage
\begin{table}[h!]
	\begin{center}
		\begin{tabular}{ |c|c|c|c| } 
			\hline
			Scale & TYP & UNIT \\
			\hline
			\verb|AFS_SEL=0| & $\pm2$ & g \\ 
			\verb|AFS_SEL=1|& $\pm4$ & g \\ 
			\verb|AFS_SEL=2|& $\pm8$ & g \\
			\verb|AFS_SEL=3|& $\pm16$ & g\\
			\hline
		\end{tabular}
		\caption{Full-scale Range Table of Accelerometer}
		\label{table:1}
	\end{center}
\end{table}
\begin{table}[h!]
	\begin{center}
		\begin{tabular}{ |c|c|c|c| } 
			\hline
			Scale & Sensitivity Scale Factor & UNIT \\
			\hline
			\verb|AFS_SEL=0| & 16,384 & LSB/g \\ 
			\verb|AFS_SEL=1|& 8,192 & LSB/g \\ 
			\verb|AFS_SEL=2|& 4,096 & LSB/g \\
			\verb|AFS_SEL=3|& 2,048 & LSB/g\\
			\hline
		\end{tabular}
		\caption{Sensitivity Scale Factor Table of Accelerometer}
		\label{table:1}
	\end{center}
\end{table}
With \verb|AFS_SEL| being bits of the accelerometer configuration register of the sensor. These bits are used to select the full scale range of the accelerometer. \par 

 The gyroscope integrated on the same board can measure the angular velocity at 4 programmable full-scale ranges ($\pm250^{o}/s$, $\pm500^{o}/s$, $\pm1000^{o}/s$ and $\pm2000^{o}/s$). This sensor has an external sync signal which supports image, video capturing and GPS synchronization. Its normal operating current is \SI{500}{\micro\ampere} which enables the reduction on power consumption. The raw output of the gyroscope is displayed in form of LSB/($^{o}/s$). Divided with an appropriate Sensitivity Scale Factor (\gls{ssf}), we can obtain the angular velocity with unit of ($^{o}/s$). Each full scale has its own SFF. Table 2.3 shows different full-scale of the gyroscope and table 2.4 shows the \gls{ssf} for each scale. 
\begin{table}[h!]
	\begin{center}
		\begin{tabular}{ |c|c|c|c| } 
			\hline
			Scale & TYP & UNIT \\
			\hline
			\verb|FS_SEL=0| & $\pm250$ & $^{o}/s$\\ 
			\verb|FS_SEL=1|& $\pm500$ & $^{o}/s$ \\ 
			\verb|FS_SEL=2|& $\pm1000$ & $^{o}/s$ \\
			\verb|FS_SEL=3|& $\pm2000$ & $^{o}/s$\\
			\hline
		\end{tabular}
		\caption{Full-scale Range Table of Gyroscope}
		\label{table:1}
	\end{center}
\end{table}
\begin{table}[h!]
	\begin{center}
		\begin{tabular}{ |c|c|c|c| } 
			\hline
			Scale & Sensitivity Scale Factor & UNIT \\
			\hline
			\verb|FS_SEL=0| & 131 & LSB/($^{o}/s$) \\ 
			\verb|FS_SEL=1|& 65.5 & LSB/($^{o}/s$) \\ 
			\verb|FS_SEL=2|& 32.8 & LSB/($^{o}/s$) \\
			\verb|FS_SEL=3|& 16.4 & LSB/($^{o}/s$)\\
			\hline
		\end{tabular}
		\caption{Sensitivity Scale Factor Table of Gyroscope}
		\label{table:1}
	\end{center}
\end{table}
\newpage
With \verb|FS_SEL| being bits of the gyroscope configuration register on the sensor. These bits are used to select the full scale range of gyroscope. \par 
After getting acceleration and angular velocity in three different directions from two sensors, normalized acceleration and normalized angular velocity are calculated with formulas (3.1) and (3.2) and compared with thresholds to detect the fall. 
\subsection{Microcontroller}
To perform all data processing and communications, the ESP8266EX \gls{mcu} integrated with L106 32-bit RISC processor from Tensilica is chosen. This \gls{mcu} archives extra low power consumption and reaches a maximum clock speed up to 160 Mhz, which is specially designed for mobile and \gls{iot} applications. The ESP8266EX \gls{mcu} supports most of popular interfaces such as GPIO, UART, I$^{2}$C, ADC, PWM etc. to connect with external peripheral devices. The figure 4.3 shows the functional block diagram of ESP8266EX. \par 
\begin{figure}[h]
	\centering
	\includegraphics[scale=0.5]{esp8266}
	\caption{ESP8266 D1 Mini Board with an ESP8266EX MCU inside}
\end{figure}
\cleardoublepage
\begin{figure}[h]
	\centering
	\includegraphics[scale=0.8]{functional_diagram}
	\caption{Functional Block Diagram of ESP8266EX MCU}
\end{figure}
The ESP8266EX \gls{mcu} integrates memory units including one 50kB SRAM and one 16MB external SPI flash that are sufficient for most \gls{iot} applications. The SRAM size is less than 50kB when the \gls{mcu} is working under Station Mode or connecting to a router. All user programs are stored in the external SPI flash and non-volatile after switching off the power.
\subsection{Wireless Module}
ESP8266EX \gls{mcu} integrates a Wi-Fi module which is implemented TCP/IP protocol with the full 802.11 b/g/n WLAN MAC standard and Wi-Fi Direct specification. The ESP8266EX is designed with advance power management technologies. The low-power architecture operates in three modes: active mode, sleep mode and deep-sleep mode. ESP8266 consumes only\SI{20}{\micro\ampere} in deep-sleep mode and the standby power consumption below 1.0mW. These two features which make the ESP8266 Wi-Fi module fully compatible with mobile wireless applications that require low power dissipation. \par
In this project, a standard TCP/IP protocol has been chosen because of its highly reliable connection and its ability to control data congestion. Additionally, this system also requires that the received data on the monitoring computer must be in the same order as the original one on sensor module. Therefore, TCP/IP is chosen because of its capability of ordering the received data and error recovery. Whenever there is an error on the connection, all the erroneous packets are retransmitted to the destination and rearranged in the correct order.\par 
A brief description of the wireless communication system: there is a TCP server running on the monitoring computer and listening the requests from clients. The ESP8266 module (programmed in \textit{client mode}) continuously sends sensors' data to the server via a Wi-Fi connection. All data is captured by a Python program and plotted on a visual graph. The received data is also stored in a database to serve later data analysis and algorithm optimization. 
\begin{figure}[h!]
	\centering
	\includegraphics[scale=0.38]{system_schematic}
	\caption{Schematic of wireless communication system}
\end{figure} 
\section{Software}

\subsection{Embeddded Software on Sensor Node}
On the sensor node, the micro-controller board is programmed to read the data from sensors, calculate normalized parameters and send data to host computer. First, the microcontroller communicates with sensors, using I$^{2}$C protocol with the help of MPU6050 library provided by Invensense\textregistered, to set the clock source and full scale range for two sensors. It is also important to set the sensor board to wake-up mode in this step to make sure it working.  After successfully connecting, the frequency of these two sensors are set to 1kHz by configuring the digital low pass filter with the \textit{setDLPFMode} function from the library. Normally, these sensors operate at the frequency of 8kHz, which means there are 8000 output samples within one second. That is too much for the requirement of a fall detection system and usually causes congestion and lost data during the transmission from sensor node to the host computer. To avoid this problem, it is necessary to low down the internal sampling rate of two sensors . See the table for the digital low pass filter configuration. \clearpage
\begin{table}[h!]
	\begin{center}
		\begin{tabular}{ |c|c|c|c|c|c| } 
			\hline
			 & \multicolumn{2}{c|}{Accelerometer} & \multicolumn{2}{c|}{Gyroscope} &  \\
			 \cline{2-5}
			 \verb|DLPF_CFG| & Bandwidth & Delay & Bandwidth & Delay & Sample Rate\\ 
			 \hline
			 0 & 260Hz & 0ms & 256Hz & 0.98ms &8kHz \\
			 \hline
			 1 & 184Hz & 2.0ms & 188Hz & 1.9ms & 1kHz\\
			 \hline
			 2 & 94Hz & 3.0ms & 98Hz & 2.8ms & 1kHz\\
			 \hline
			 3 & 44Hz & 4.9ms & 42Hz & 4.8ms & 1kHz\\
			 \hline
			 4 & 21Hz & 8.5ms & 20Hz & 8.3ms & 1kHz\\
			 \hline
			 5 & 10Hz & 13.8ms & 10Hz & 13.4ms & 1kHz\\
			 \hline
			 6 & 5Hz & 19.0ms & 5Hz & 18.6ms & 1kHz\\
			 \hline
		\end{tabular}
		\caption{Sensitivity Scale Factor Table of Gyroscope}
	\end{center}
\end{table}
Next, like other sensors, both gyroscope and accelerometer always have non-zero errors even when leveling and need to be calibrated with a set of offset values before being used. These offset are calculated by another program included in the MPU6050 library and set into appropriate sensor's registers with provided function from the library. See the appendix C for the register map of the MPU6050 sensor. \par
After initializing the sensors, the sensor node establishes a wireless connection with the local network and host computer. While the host is not connected, the sensor node continuously calls for a connection and the led indicator on the micro-controller board keep brighting until receiving a success connection. Once the connection is established, the sensor node turns off the indicator led for a while and then start blinking while transferring data. All the connection handling functions are included in the \textit{Connection Handling} library written by myself. See the Appendix B.\par 

Finally, the micro-controller starts reading data from two sensors. The output data is used to calculate the normalized acceleration and angular velocity. Whenever the normalized acceleration falls below the $LFT_{Acc}$ threshold, the first flag is set true to indicate the start of a fall event and a 500ms timer starts counting. For the next 500ms, if normalized acceleration and angular velocity exceed the $UFT_{Acc}$ and $UFT_{Gyro}$ respectively, the second and third flags are also set true. When the timer is executed, all flags are checked. If three flags are all true, a fall is detected. The sensor node will send an emergency email to the caregivers while start blinking the super bright red led simultaneously, that help the caregivers find out the patients when they reach the patients' location. If one condition is not true, the system start a new detection loop. All the reading data are also converted into strings, concatenated and sent to the host computer for monitoring. See the Appendix A for the main code on the sensor node. 
\subsection{Data Acquisition Software}

\subsection{Data Analysis Software}

\section{System Intergration}

\chapter{Experimental and Procedure}

\section{Data Collection}

\section{Data Analysis}

\section{Experiment on Real Subjects to Evaluate the Performance of The Network}

\chapter{Results and Discussion}

\section{Results Assessment}

\section{Comparing the performance with existing works}

\chapter{Conclusion and Future Works}

\section{Summary}

\section{Limitations}

\section{Future works}
%======================================================================


\appendix
% Add a title page before the appendices and a line in the Table of Contents
\addcontentsline{toc}{chapter}{APPENDICES}
%======================================================================
\chapter{C/C++ Code on the Sensor Node}
\begin{lstlisting}
	#include <Event.h>
	#include <Timer.h>
	#include <AsyncPrinter.h>
	#include <async_config.h>
	#include <ESPAsyncTCP.h>
	#include <ESPAsyncTCPbuffer.h>
	#include "MPU6050.h"
	#include "Wire.h"
	#include "WiFiClient.h"
	#include "ESP8266WiFiMulti.h"
	#include "ESP.h"
	#include "Time.h"
	#include <CSensorSender.h>
	
	
	ESP8266WiFiMulti WiFiMulti ;
	#if I2CDEV_IMPLEMENTATION == I2CDEV_ARDUINO_WIRE
	#include "Wire.h"
	#endif
	#define OUTPUT_READABLE_ACCELGYRO
	//#define SERIAL_DEBUG
	#define BUZZER 13
	#define CONN_INTERVAL     (1000)
	#define NUMB_ELEMS_PACKET (100)
	#define SAMPL_INTV        (CONN_INTERVAL/NUMB_ELEMS_PACKET)
	#define dataRate 9
	#define lowAcc 0.26
	#define highAcc 2.75
	#define highGyro 254.5
	
	Timer t;
	const uint16_t port = 8000;
	String host = "192.168.137.1"; // ip or dns
	
	CSensorSender mySender(host, port, NUMB_ELEMS_PACKET, 2);
	
	// Use WiFiClient class to create TCP connections
	AsyncClient client;   
	MPU6050 accelgyro;
	
	//MPU6050 accelgyro(0x69); // <-- use for AD0 high
	int16_t ax, ay, az;
	int16_t gx, gy, gz;
	float t_time = 0,send_time;
	float rotX, rotY, rotZ,normAcc;
	float gyroX, gyroY, gyroZ, normGyro;
	String packet = "",packetSend = "";
	int8_t i=0;
	bool blinkState = false;
	bool firstCond = false, secCond = false, thirdCond = false;
	
	void Print_IMU()
	{
			#ifdef SERIAL_DEBUG
			Serial.println("Accelerometer:");
			Serial.print(ax);
			Serial.print("\t");
			Serial.print(ay);
			Serial.print("\t");
			Serial.println(az);
			Serial.println("Gyroscope: ");
			Serial.print(gx);
			Serial.print("\t");
			Serial.print(gy);
			Serial.print("\t");
			Serial.println(gz);
			#endif
			Serial.print(normAcc);
			Serial.print(" ");
			Serial.println(normGyro);
	}
	void alert(){
			if (firstCond == true && secCond == true && thirdCond == true)
			{
				digitalWrite(BUZZER, HIGH);
			}
			firstCond = false;
			secCond = false;
			thirdCond = false;
	}
	void init_IMU() {
			// join I2C bus (I2Cdev library doesn't do this automatically)
			Wire.begin();
			Serial.begin(115200);
			// initialize device
			Serial.println("Initializing I2C devices...");
			accelgyro.initialize();
			// verify connection
			Serial.println("Testing device connections...");
			Serial.println(accelgyro.testConnection() ? "MPU6050 connection successful" : "MPU6050 connection failed");
			Serial.print("\n");
			accelgyro.setDLPFMode(1); //set the Frequency to 1kHz
			accelgyro.setRate(dataRate); //Set data Rate: Rate = (Freq)/(dataRate+1)
			Serial.print("DLPF Mode: ");
			Serial.println(accelgyro.getDLPFMode());
			Serial.print("Data rate of sensor: ");
			Serial.println(accelgyro.getRate());
			// use the code below to change accel/gyro offset values
			Serial.println("Updating internal sensor offsets...");
			accelgyro.setXGyroOffset(55);
			accelgyro.setYGyroOffset(-28);
			accelgyro.setZGyroOffset(-2);
			accelgyro.setXAccelOffset(-2581);
			accelgyro.setYAccelOffset(-3937);
			accelgyro.setZAccelOffset(1199);
	}
	void Init_Wifi()
	{
			// Initialize Wifi_connection
			WiFiMulti.addAP("MinhKhang", "qwertyuiop");
			Serial.println();
			Serial.println();
			Serial.print("Wait for WiFi... ");
			while(WiFiMulti.run() != WL_CONNECTED) {
			Serial.print(".");
			delay(500);
			ESP.wdtFeed();
			}
			Serial.println("");
			Serial.println("WiFi connected");
			Serial.println("IP address: ");
			Serial.println(WiFi.localIP());
			digitalWrite(2, LOW);
			delay(500);
	}
	
	void setup(){
			//initialize pin mode
			pinMode(BUZZER, OUTPUT);
			Serial.begin(115200);
			Wire.begin();
			init_IMU();
			Init_Wifi();
	}
	void loop() {
			sensorData_t;
			ESP.wdtFeed();
			if ((millis()-t_time)>SAMPL_INTV)
			{ 
				//get data from sensor
				accelgyro.getMotion6(&ax, &ay, &az, &gx, &gy, &gz);
				rotX = ax / 16384.0;
				rotY = ay / 16384.0;
				rotZ = az /16384.0;
				normAcc = sqrt(rotX*rotX + rotY*rotY + rotZ*rotZ);
				gyroX = gx /131.0;
				gyroY = gy /131.0;
				gyroZ = gz /131.0;
				normGyro = sqrt(gyroX*gyroX + gyroY*gyroY + gyroZ*gyroZ);
				//Start the comparator
				if (normAcc < lowAcc)
				{
					firstCond = true;
					t.after(500,alert);
				}
				if (firstCond == true && normAcc > highAcc)
				{
					secCond = true;
				}
				if (firstCond == true && normGyro > highGyro)
				{
					thirdCond =true;
				}
				t_time=millis();
				currentSensorData.normAccel = normAcc;
				currentSensorData.normGyro = normGyro;
				mySender.queueSensorData(currentSensorData);
			}
			t.update();
	}
\end{lstlisting}%

\chapter{Connection Handling Library}

\begin{lstlisting}
	#include "CSensorSender.h"
	
	#define DEBUG_PRINT 1
	
	static inline void debugPrint(String inputString)
	{
		#if DEBUG_PRINT
		Serial.println(inputString);
		#endif
	}
	
	void clientConnectedHandler(void* arg, AsyncClient* aClient)
	{
		debugPrint("Connected");
		String sentString = ((CSensorSender*)arg)->getSentString();
		aClient->write(sentString.c_str());
		((CSensorSender*)arg)->setIndicatorLed(LOW);
	}
	
	void clientDisconnectedHandler(void* arg, AsyncClient* aClient)
	{
		debugPrint("Disconnected");
		senderState_t senderState = ((CSensorSender*)arg)->getSenderState();
		if ((senderState == SENDING) || (senderState == SENDING_ERROR)) {
			((CSensorSender*)arg)->setSenderState(READY);
		}
		((CSensorSender*)arg)->setIndicatorLed(HIGH);
	}
	
	void clientErrorHandler(void* arg, AsyncClient* aClient, unsigned char error)
	{
		debugPrint("Error");
		senderState_t senderState = ((CSensorSender*)arg)->getSenderState();
		if (senderState == SENDING) {
		((CSensorSender*)arg)->setSenderState(SENDING_ERROR);
		}
		((CSensorSender*)arg)->setIndicatorLed(HIGH);
	}
	
	void CSensorSender::populateSentString()
	{
		mSentString = "";
		for (int i = 0; i < mNoOfSamples; i++) {
		mSentString += String(mSensorData[i].normAccel) + "," + String(mSensorData[i].normGyro);
			if (i < mNoOfSamples - 1) {
				mSentString += ",";
			}
			else {
				mSentString += "\n";
			}
		}
	}
	
	// Constructor
	CSensorSender::CSensorSender(String IPString, int16_t port, uint16_t noOfSamplesPerPacket, int ledIndicatorPin)
	{
		// Load the arguments onto the attributes
		mIPString = IPString;
		mPort = port;
		mNoOfSamplesPerPacket = noOfSamplesPerPacket;
		// Allocate the exact number of elements of sensorData_t
		// this is going to be the array to store queued data
		mSensorData = new sensorData_t[mNoOfSamplesPerPacket];
		mSentString = "";
		mNoOfSamples = 0;
		// Initialize the state to READY
		mSenderState = READY;
		mLedIndicatorPin = ledIndicatorPin;
		pinMode(ledIndicatorPin, OUTPUT);
		digitalWrite(ledIndicatorPin, HIGH);
		// Initialize the ASyncClient
		mClient.onConnect(clientConnectedHandler, this);
		mClient.onError(clientErrorHandler, this);
		mClient.onDisconnect(clientDisconnectedHandler, this);
	}
	
	senderErrorCode_t CSensorSender::queueSensorData(sensorData_t& sensorDataInput)
	{
		senderErrorCode_t returnSenderErrorCode = SENDER_ERROR_NOT_READY;
		// If there is still space left in the buffer
		if (mNoOfSamples < mNoOfSamplesPerPacket) {
		mSensorData[mNoOfSamples] = sensorDataInput;
		mNoOfSamples++;
		returnSenderErrorCode = SENDER_ERROR_OK;
	}
	else if (mSenderState == READY) {
		// Update the state
		mSenderState = SENDING;
		// Since this is a normal behavior
		returnSenderErrorCode = SENDER_ERROR_OK;
		populateSentString();
		mNoOfSamples = 0;
		debugPrint(mSentString);
		//Do the sending here
		mClient.connect(mIPString.c_str(), mPort);
	}
	else {
		returnSenderErrorCode = SENDER_ERROR_NOT_READY;
	}
	return returnSenderErrorCode;
	}
	
	// Destructor, clean up the allocated memory here
	CSensorSender::~CSensorSender()
	{
		// Deallocated allocated memory
		delete[] mSensorData;
	}
	
	String CSensorSender::getSentString()
	{
		return mSentString;
	}
	
	senderState_t CSensorSender::getSenderState()
	{
		return mSenderState;
	}
	
	void CSensorSender::setSenderState(senderState_t senderState)
	{
		mSenderState = senderState;
	}
	
	void CSensorSender::setIndicatorLed(uint8_t ledState)
	{
		digitalWrite(mLedIndicatorPin, ledState);
	}
\end{lstlisting}
\chapter{Register Map of MPU6050}
\begin{figure}[h!]
	\includegraphics{5}
\end{figure} 
\begin{figure}[h!]
	\includegraphics{6}
\end{figure}
\begin{figure}[h!]
	\includegraphics{7}
\end{figure}
\chapter{Python Code for Data Acquisition Program}
\begin{verbatim}
import matplotlib.pyplot as plt
import matplotlib.animation as animation
from drawnow import *
import socketserver

accel = []
gyro = []
plt.ion() #tell the matplotlib that we want to draw live data
def makeFig():
plt.subplot(2,1,1)
plt.plot(accel, 'r', label= 'Normalized Acceleration')
plt.legend(loc='upper left')
plt.ylim(0,5)
plt.subplot(2,1,2)
plt.plot(gyro, 'b', label = 'Normalized Angular Velocity')
plt.legend(loc='upper left')
plt.savefig('testplot.png')

class myTCPServer(socketserver.StreamRequestHandler):
def handle(self):
data = self.rfile.readline()
dataArray = data.decode().split(',')
normAcc = float(dataArray[0])
normGyro = float(dataArray[1])
accel.append(normAcc)
gyro.append(normGyro)
if len(accel) > 100:
accel.pop(0)
if len(gyro) > 100:
gyro.pop(0)
drawnow(makeFig)
#create TCP server
serv = socketserver.TCPServer(("",8000),myTCPServer)
serv.serve_forever()
\end{verbatim}%
%----------------------------------------------------------------------
% END MATERIAL
%----------------------------------------------------------------------

% B I B L I O G R A P H Y
% -----------------------

% The following statement selects the style to use for references.  It controls the sort order of the entries in the bibliography and also the formatting for the in-text labels.

% This specifies the location of the file containing the bibliographic information.  
% It assumes you're using BibTeX (if not, why not?).
\cleardoublepage % This is needed if the book class is used, to place the anchor in the correct page,
                 % because the bibliography will start on its own page.
                 % Use \clearpage instead if the document class uses the "oneside" argument
\phantomsection  % With hyperref package, enables hyperlinking from the table of contents to bibliography             
% The following statement causes the title "References" to be used for the bibliography section:
\renewcommand*{\bibname}{References}

% Add the References to the Table of Contents
\addcontentsline{toc}{chapter}{\textbf{References}}
\bibliographystyle{IEEEtran}
\bibliography{uw-ethesis}
% Tip 5: You can create multiple .bib files to organize your references. 
% Just list them all in the \bibliogaphy command, separated by commas (no spaces).

% The following statement causes the specified references to be added to the bibliography% even if they were not 
% cited in the text. The asterisk is a wildcard that causes all entries in the bibliographic database to be included (optional).


\end{document}
